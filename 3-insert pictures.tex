\documentclass[12pt,letterpaper]{article}
% 12pt:the font size, default size is 10pt

%letterpaper: the paper size,default is A4
\usepackage[utf8]{inputenc}
% the encoding for the document
\usepackage{comment}
% add '\usepackage{comment}' to the preamble when use muti-line comments instead of putting a % at the beginning of each line.

% use \% when you actually need to print this symbols

\usepackage{graphicx}
\graphicspath{{./images/}}
% the images are kept in a folder named images

\title{First document}
% the title of the document is First document

\author{Hubert Farnsworth \thanks{funded by the Overleaf team}}

% \author{aa}:put the names of the authors
% thanks{}: an optional parameter;it will add a superscript and a footnote with the text inside the braces.
\date{February 2014}

\begin{document}


\begin{titlepage}
% declares an block of code. whatever you include in titlepage will appear in the first page of your document
\maketitle
% print the title,the author and the date. Without the code, your title ,author and the date will not show in the document.


\begin{abstract}
This is a simple paragraph at the beginning of the document. A brief introduction to the main subject.
\end{abstract}
% the abstract environment will put the text in a special format at the top of the document.

\end{titlepage}

In this document some extra packages and parameters
were added. There is an encoding package
and pagesize and fontsize parameters.


This line will start a second paragraph. And I can
 break\\ the lines \\ and continue on a new line.

\begin{comment}
This text won't show up in the compiled pdf
this is just a multi-line comment. Useful
to, for instance, comment out slow-rendering parts
while working on a draft.
\end{comment}

% new paragraph:hit the 'enter' key twice
% start a new line:\\


\includegraphics{universe}
% universe is the name of the file containing the image without the extension

\includegraphics[scale=1.5]{lion-logo}
% change the size of the image:scale the image 1.5 of its real size

\includegraphics[width=3cm,height=4cm]{lion-logo}

% sacle the image to a some specific width and height.

\begin{comment}
% Images: captions,labels, and references
3 important commands: 
\caption{a nice plot}:set the caption for the figure.You can place it above or below the figure. Above: \caption{} first,then \includegraphics[]{}. Below: \includegraphics[]{} first, then \caption{}.
\label{fig:mesh1}:If you need to refer the image within your document,set a label with this command.The label will number the image.
\ref{fig:mesh1}: Refer the image in your document. It will be substituted by the number corresponding to the referenced figure.

\end{comment}

\begin{figure}[h]
    \caption{a nice plot}
    \includegraphics[width=0.25\textwidth]{mesh}
    \centering
    \label{fig:mesh1}
\end{figure}

As you can see in the figure \ref{fig:mesh1},the function grows near 0.
Also,in the page \pageref{fig:mesh1} is the same example.

% \pageref{fig:mesh1}: This prints out the page number where the referenced image appears.

% reset the pictures of lion and universe
\begin{figure}[h]
    \centering
    \includegraphics[width=10cm,height=8cm]{universe}
    \caption{This is a picture of universe}
    \label{fig:universe}
\end{figure}

Above,we use the universe picture \ref{fig:universe} as a practice. It shows in the page \pageref{fig:universe}.

\begin{figure}[h]
    \caption{This is a picture of universe}
    \centering
    \includegraphics[width=10cm,height=8cm]{lion-logo}
    \label{fig:lion}
\end{figure}
We use the lion-logo\ref{fig:lion} as the second practice. It shows in the page \pageref{fig:lion}

% show all the list in the document:
\listoffigures
\end{document}
