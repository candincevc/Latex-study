% Adding math to latex

\documentclass[12pt,A4]{article}

\usepackage[utf8]{inputenc}

\usepackage{amsmath}
% Many math mode commands require the amsmath package, so be sure to include it when writing math.

\title{Adding math to Latex}
\date{2020-12-16}

\begin{titlepage}

\maketitle

\begin{document}

\textbf{1 writing modes for mathematical expressions:}
\begin{itemize}
    

\item \underline{\textbf{the inline mode}}: be used to write formulas that are part of a text.

\item \underline{\textbf{the display mode}}: be used to write expressions that are put on separate lines.

\end{itemize}

1.1 The inline mode

In physics, the mass-energy equivalence is stated by the equation $E=mc^2$,discovered in 1905 by Albert Einstein.

1.2 The displayed mode

The displayed mode has two versions: numbered and unnumbered.

The mass-energy equivalence is described by the famous equation
\[E=mc^2\]
discovered in 1905 by Albert Einstein. In natural units ($c=1$),the formula express the identity
\begin{equation}
    E=m
\end{equation}
\begin{equation}
    E=mc*c
\end{equation}


\textbf{2 Other math mode commands}

(1) Subscripts in math mode are written as $a_b$.

(2) Superscripts are written as $a^b$.

(3) Combined subscripts and superscripts

\[T^{i_1 i_2 \dots i_p}_{j_1 j_2 \dots j_q}=
T(X^{i_1},\dots,x^{i_p},e_{j_1},\dots,e_{j_q})
\]

(4) Integrals

using $\int$

(5) fractions

using $\frac{a}{b}$

(6) limits

\[\int_0^1 \frac{dx}{e^x}=\frac{e-1}{e} \]

(7) geek letters

$\omega$ $\delta$ $\Omega$ $\Delta$

(8) mathematical operators

$\sin(\beta)$ $\cos(\alpha)$ $\log(x)$


\newpage

\textbf{3 Matrices}

% First, you need to import \usepackage{amsmath}

(1) plain

\begin{matrix}
1 & 2 & 3 \\
a & b & c
\end{matrix}

(2) Parentheses; round brackets

\begin{pmatrix}
1 & 2 & 3 \\
a & b & c

\end{pmatrix}


(3) brackets;square brackets

\begin{bmatrix}
1 & 2 & 3\\
a & b & c
\end{bmatrix}

(4) braces
\begin{bmatrix}
a & b & c \\
1 & 2 & 3
\end{bmatrix}

(5)pipes

\begin{vmatrix}
a & b & c \\
1 & 2 & 3
\end{vmatrix}

(6) double pipes

\begin{vmatrix}
a & b & c \\
1 & 2 & 3
\end{vmatrix}

\end{document}

\end{titlepage}